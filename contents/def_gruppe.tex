%!TEX root = ../trajectory-grouping.tex

Wir bezeichnen mit $\mathcal{X}$ die Menge der Entitäten, zu denen uns der Ort innerhalb eines gewissen Zeitraums bekannt ist.
Wie in \cref{cha:einleitung} angedeutet wird unsere Definition einer Gruppe von den folgenden drei Parametern abhängen, die Charakteristika der zu betrachtenden Gruppen bestimmen:
\begin{itemize}
	\item Räumlicher Parameter $\varepsilon > 0$
	\item zeitlicher Parameter $\delta > 0$
	\item Mindestgröße $m \in \mathbb{N}\setminus \set*{0}$
\end{itemize}
Wir beginnen mit einigen Definitionen zum räumlichen Zusammenhang zu einem festen Zeitpunkt $t$.
\begin{definition}[{name=[Epsilon-Zusammenhang]}]
	Die $\varepsilon$-Scheibe $B_\varepsilon^t(x)$ einer Entität $x$ zum Zeitpunkt $t$ ist eine Scheibe mit Radius $\varepsilon$ um $x$ zu Zeitpunkt $t$, also $B_\varepsilon^t(x) = \set*{y \in \mathbb{R}^d \given \norm*{x^t-y} \le \varepsilon}$.\marginnote{wir betrachten meist $d=2$}
	
	Zwei Entitäten $x$ und $y$ sind zum Zeitpunkt $t$ \Index{direkt zusammenhängend}, falls sich die $\varepsilon$-Scheiben überlappen, das heißt der Schnitt der beiden offenen Scheiben nicht leer ist.
	
	Die Entitäten $x$ und $y$ heißen \bet{$\varepsilon$-zusammenhängend}\index{epsilon-zusammenhängend@$\varepsilon$-zusammenhängend} zum Zeitpunkt $t$, falls eine Folge von Entitäten $x=x_0, \ldots, x_k = y$ existiert, sodass $x_i$ und $x_{i+1}$ direkt zusammenhängend sind.
\end{definition}

Eine Teilmenge $\mathcal{S} \subseteq \mathcal{X}$ heißt dementsprechend \bet{$\varepsilon$-zusammenhängend}, falls alle Entitäten in $\mathcal{S}$ paarweise $\varepsilon$-zusammenhängend sind.
Dies ist äquivalent dazu, dass das Innere der Vereinigung aller $\varepsilon$-Scheiben der Entitäten von $\mathcal{S}$ zusammenhängend ist.
Wir nennen $\mathcal{S}$ eine \Index{Komponente} zum Zeitpunkt $t$, falls $\mathcal{S}$ eine maximale $\varepsilon$-zusammenhängende Teilmenge ist.
Die Menge der Komponenten zum Zeitpunkt $t$ bezeichnen wir mit $\mathcal{C}(t)$; sie bildet eine Partition von $\mathcal{X}$ zum Zeitpunkt $t$.

\Cref{fig:components} zeigt die Auswirkung verschiedener $\varepsilon$-Werte auf die Partition der Entitäten in verschiedene $\varepsilon$-Komponenten.

\begin{figure}[bthp]
	\Centering
	\subfloat[Mit $\varepsilon=\SI{0.4}{\centi\metre}$ liegen alle Punkte in einer Komponente]{
		\begin{tikzpicture}[scale=1.2]
			% \draw[help lines] (-1,-1) grid (2,2);
			\coordinate (1) at (0,-0.02);
			\coordinate (2) at (.05,.4);
			\coordinate (3) at (.5,.2);
			\coordinate (4) at (.85,-.08);
			\coordinate (5) at (1.45,-.04);
			\coordinate (6) at (-0.7,.3);
			\coordinate (7) at (2.1,.3);
			\coordinate (8) at (2.05,1);
			\coordinate (9) at (0.2,1.1);
			\coordinate (10) at (.9,-.65);
			\coordinate (11) at (2.2,.6);
			\coordinate (12) at (-.3,1);
			\coordinate (13) at (-.8,.9);
			\coordinate (14) at (-1.1,.5);
			\foreach \x in {1,2,...,14}{
				\draw[DodgerBlue3,fill=DodgerBlue3,fill opacity=.35,very thick] (\x) circle[radius=.4];
				\draw[fill,very thick] (\x) circle[radius=.02];
			}
		\end{tikzpicture}
	}
	\hspace{2cm}
	\subfloat[Mit $\varepsilon=\SI{0.3}{\centi\metre}$ erhält man drei Komponenten]{
		\begin{tikzpicture}[scale=1.2]
			% \draw[help lines] (-1,-1) grid (2,2);
			\coordinate (1) at (0,-0.02);
			\coordinate (2) at (.05,.4);
			\coordinate (3) at (.5,.2);
			\coordinate (4) at (.85,-.08);
			\coordinate (5) at (1.45,-.04);
			\coordinate (6) at (-0.7,.3);
			\coordinate (7) at (2.1,.3);
			\coordinate (8) at (2.05,1);
			\coordinate (9) at (0.2,1.1);
			\coordinate (10) at (.9,-.65);
			\coordinate (11) at (2.2,.6);
			\coordinate (12) at (-.3,1);
			\coordinate (13) at (-.8,.9);
			\coordinate (14) at (-1.1,.5);
			\foreach \x in {1,2,...,5,10}{
				\draw[DodgerBlue3,fill=DodgerBlue3,fill opacity=.35,very thick] (\x) circle[radius=.3];
				\draw[fill,very thick] (\x) circle[radius=.02];
			}
			\foreach \x in {6,9,12,13,14}{
				\draw[Firebrick2,fill=Firebrick2,fill opacity=.35,very thick] (\x) circle[radius=.3];
				\draw[fill,very thick] (\x) circle[radius=.02];
			}
			\foreach \x in {7,8,11}{
				\draw[SeaGreen3,fill=SeaGreen3,fill opacity=.35,very thick] (\x) circle[radius=.3];
				\draw[fill,very thick] (\x) circle[radius=.02];
			}
		\end{tikzpicture}
	}
	\caption{Komponenten einer 14-elementigen Punktmenge zu verschiedenen $\varepsilon$-Werten}\label{fig:components}
\end{figure}

\begin{definition}[{name=[Gruppe während eines Zeitraums]},label=def:gruppe]
	Eine Menge $G$ von $k$ Entitäten bildet eine \Index{Gruppe} im Zeitintervall $I$, falls folgendes gilt:
	\begin{enumerate}[(i)]
		\item $G$ enthält mindestens $m$ Entitäten, also $k \ge m$
		\item die Länge von $I$ ist mindestens $\delta$
		\item\label{enum:3:def:gruppe}zu jeder Zeit $t \in I$ existiert eine Komponente $C \in \mathcal{C}(t)$ mit $G \subseteq C$.\marginnote{in \textcite{grouping_improved} wird zusätzlich gefordert, dass alle $x,y \in G$ \emph{in} $G$ $\varepsilon$"=zusammenhängend sind}
	\end{enumerate}
	Das Intervall $I=[t_s,t_e]$ der Gruppe $G$ bezeichnen wir auch mit $I_G$.
	Eine Gruppe $H$ \bet{überlagert}\index{Gruppe!überlagernde} $G$, falls $G \subseteq H$ und $I_G \subseteq I_H$ gilt.
	Falls kein solches $H$ existiert, so heißt $G$ \bet{maximal}\index{Gruppe!maximale}.
\end{definition}

\begin{figure}[thbp]
	\Centering
	\begin{tikzpicture}[scale=1.3]
		% Zeitpunkte definieren
		\coordinate (t_0) at (0,0);
		\coordinate (t_1) at (2,0);
		\coordinate (t_2) at (3.2,0);
		\coordinate (t_3) at (4.5,0);
		\coordinate (t_4) at (5.7,0);
		\coordinate (t_5) at (6.2,0);
		
		\draw[gray,very thick] ($(t_0) - (.2,0)$) -- ($(t_5) + (.2,0)$);
		\foreach \x in {t_0,t_1,t_2,t_3,t_4,t_5}{
			\draw[gray, thick] ($(\x) + (0,.1)$) -- ($(\x)- (0,.1)$) node[below]{$\x$};
		}
		
		% Trajektorien
		\coordinate (0x_1) at ($(t_0) + (0,3)$);
		\coordinate (1x_1) at ($(t_1) + (0,2.7)$);
		\coordinate (2x_1) at ($(t_2) + (0,2.9)$);
		\coordinate (3x_1) at ($(t_3) + (0,2.85)$);
		\coordinate (4x_1) at ($(t_4) + (0,3.7)$);
		\coordinate (5x_1) at ($(t_5) + (0,3.4)$);
		               
		\coordinate (0x_2) at ($(t_0) + (0,2.8)$);
		\coordinate (1x_2) at ($(t_1) + (0,2.6)$);
		\coordinate (2x_2) at ($(t_2) + (0,2.8)$);
		\coordinate (3x_2) at ($(t_3) + (0,2.77)$);
		\coordinate (4x_2) at ($(t_4) + (0,2.35)$);
		\coordinate (5x_2) at ($(t_5) + (0,2.4)$);
		               
		\coordinate (0x_3) at ($(t_0) + (0,.8)$);
		\coordinate (1x_3) at ($(t_1) + (0,1)$);
		\coordinate (2x_3) at ($(t_2) + (0,2.7)$);
		\coordinate (3x_3) at ($(t_3) + (0,2.68)$);
		\coordinate (4x_3) at ($(t_4) + (0,2.2)$);
		\coordinate (5x_3) at ($(t_5) + (0,2.35)$);
		               
		\coordinate (0x_4) at ($(t_0) + (0,.65)$);
		\coordinate (1x_4) at ($(t_1) + (0,.9)$);
		\coordinate (2x_4) at ($(t_2) + (0,1.3)$);
		\coordinate (3x_4) at ($(t_3) + (0,.4)$);
		\coordinate (4x_4) at ($(t_4) + (0,2.05)$);
		\coordinate (5x_4) at ($(t_5) + (0,2.28)$);
		
		\foreach \x/\y in {x_1/DodgerBlue3,x_2/Firebrick2,x_3/SeaGreen3,x_4/Sienna2}{
			\draw[very thick,\y] (0\x) node[left=1.4ex]{$\x$} -- (1\x) -- (2\x) -- (3\x) -- (4\x) -- (5\x);
		}
		
		% epsilon-discs
		\foreach \x in {x_1,x_2,x_3,x_4}{
			\foreach \n in {0,1,...,5}{
				\draw[darkgray,fill=gray,fill opacity=.2] (\n\x) circle[radius=.107];
			}
		}
	\end{tikzpicture}
	\caption[Trajektorien mit eingezeichneten $\varepsilon$-Scheiben]{Trajektorien zu vier Entitäten $x_1, \ldots ,x_4$ mit eingezeichneten $\varepsilon$-Scheiben.
	Die maximalen Gruppen sind $\set*{x_1,x_2}$ während $[t_0,t_3]$, $\set*{x_1,x_2,x_3}$ während $[t_2,t_3]$, $\set*{x_2,x_3}$ während $[t_2,t_5]$, $\set*{x_3,x_4}$ während $[t_0,t_1]$ und $\set*{x_2,x_3,x_4}$ während $[t_4,t_5]$. Hier wurde $m=2$ und $\delta < \abs*{t_5 -t_4}$ gewählt. Wählt man stattdessen $\abs*{t_1-t_2}\ge \delta > \abs*{t_5 -t_4}$, so fällt die Gruppe $\set*{x_2,x_3,x_4}$ weg.}\label{fig:max-groups}
\end{figure}

In \cref{fig:max-groups} ist eine Menge von Trajektorien gezeichnet, bei der keine Entität eindeutig einer maximalen Gruppe zuzuordnen ist.
Da wir ja eigentlich die Trajektorien einzelner Entitäten gruppieren wollen, mag dies auf den ersten Blick etwas kontraintuitiv erscheinen.
Tatsächlich benötigen wir aber alle diese Informationen, um später die Entwicklung der Gruppen über die Zeit und insbesondere Start-, End-, Split- und Merge-Ereignisse von Gruppen analysieren zu können.\todo{siehe Kap. 1?}

Wie \cref{fig:components} bereits zeigt, hat die Wahl des Parameters $\varepsilon$ einen großen Einfluss auf die Komponenten und es stellt sich die Frage, wie sich die (maximalen) Gruppen bei Variation dieses Parameters sowie der anderen beiden $\delta$ und $m$ verändern.

Wenn $G$ eine Gruppe auf dem Intervall $I$ ist, so bleibt sie es offensichtlich bei Verminderung der Mindestgröße $m$ und der minimalen Dauer $\delta$; des Weiteren bleibt eine maximale Gruppe maximal.
Eine ähnliche Monotonie-Eigenschaft erhalten wir auch für den räumlichen Parameter $\varepsilon$:
Vergrößert man $\varepsilon$, so entstehen potentiell zu jedem Zeitpunk weniger, aber dafür größere Komponenten, sodass \cref{enum:3:def:gruppe} aus \cref{def:gruppe} eher erfüllt ist.
Folglich gibt es zu einer Gruppe $G$ bezüglich des ursprünglichen Parameters $\varepsilon$ eine Gruppe $G'$ mit $G \subseteq G'$.

Diese Monotonie-Eigenschaften haben zur Folge, dass bei einer detaillierten Sicht -- im Sinne von \cref{cha:einleitung} -- durch Verringerung der Parameter $\delta$ und $m$ sowie Vergrößerung von $\varepsilon$ die maximalen Gruppen erhalten bleiben.
Dabei können sich Größe und Dauer einer maximalen Gruppe durch ein größeres $\varepsilon$ allerdings ausdehnen.

Wir werden im weiteren Verlauf die \Index{\GrpStruktur} bestehend aus der Menge der maximalen Gruppen und einem sogenannten \Index{Reeb-Graphen} betrachten.
Neben dem Partitionieren in Trajektorien wird es uns diese \GrpStruktur erlauben, diverse Fragen an die gegeben Daten, wie zum Beispiel die folgenden, zu beantworten:
\begin{itemize}
	\item Was ist die größte/am längsten andauernde maximale Gruppe zum Zeitpunkt $t$?
	\item Wie viele Entitäten sind momentan (nicht) in einer maximalen Gruppe?
	\item Welche maximale Gruppe endet oder startet zuerst nach dem Zeitpunkt $t$?
	\item Wie lange war eine Entität insgesamt Teil einer maximalen Gruppe?
	\item Welche Entität war mit den meisten anderen Entitäten zusammen in einer maximalen Gruppe?
\end{itemize}

Bevor wir nun im nächsten Kapitel den Reeb-Graphen definieren und damit dann die Repräsentation der \GrpStruktur inklusive ihrer Komplexität besprechen, sollte an dieser Stelle noch eine kleine Unzulänglichkeit unserer Definition von Gruppen zur Sprache kommen.
Dazu stelle man sich die folgende Situation vor: Eine Entität $x$ reist über einen langen Zeitraum zusammen mit einer Gruppe $G$, verlässt diese zwischendurch aber für einen sehr kurzen Zeitraum.
Im Widerspruch zur obigen \cref{def:gruppe} mag es nun sinnvoller sein, $G \cup \set*{x}$ als Gruppe zu betrachten.
Man möchte die Gruppen also \Index{robust} gegenüber kleinen Störungen machen, die insbesondere bei Daten aus der Praxis oftmals auftreten.
Wie sich dies unter Einbeziehung der maximalen Störungsdauer als weiteren Parameter realisieren lässt, werden wir in \cref{cha:robust} gegen Ende dieses Arbeit behandeln.
