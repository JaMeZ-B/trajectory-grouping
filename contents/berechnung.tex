%!TEX root = ../trajectory-grouping.tex
Zur Berechnung der \GrpStruktur müssen wir gemäß \cref{def:grpstrukur} den reduzierten Reeb-Graphen und die maximalen Gruppen berechnen.
Dazu berechnen wir zunächst die unreduzierten Reeb-Graphen $\mathcal{R}$ und dann die maximalen Gruppen. 
Mit diesen Informationen ist die Reduktion von $\mathcal{R}$ ein trivialer Nachbearbeitungsschritt.

\section{Berechnung des Reeb-Graphen} % (fold)
\label{sec:berechnung_reeb}
Zunächst einmal erinnern wir uns, dass wir annehmen, dass alle bekannten Positionen der einzelnen Trajektorien an den gleichen Zeitpunkten sind.
Bevor wir mit der eigentlichen Berechnung beginnen können, müssen wir alle Zeitpunkte berechnen, an denen zwei Entitäten $x$ und $y$ den Abstand $2 \varepsilon$ haben.
Diese Zeitpunkte wollen wir unterteilen in \emph{Connect}- und \emph{Disconnect}-Ereignisse, wobei bei ersteren $x$ und $y$ direkt-zusammenhängend und bei letzteren nicht länger direkt-zusammenhängend werden.
Zeitpunkte, die sich nicht sinnvoll derartig kategorisieren lassen, filtern wie heraus:

\paragraph{Ereignisse berechnen} % (fold)
\label{par:ereignisse_berechnen}
Gegeben Entitäten $x$ und $y$ wollen wir eine Liste $L_{xy}$ aller Ereignisse berechnen.
Diese soll chronologisch sortiert sein und nur echt verschiedene Zeitpunkte enthalten.
Weiter müssen sich dann notwendigerweise \emph{Connect}- und \emph{Disconnect}-Ereignisse stets abwechseln.

Dazu betrachten wir in zeitlicher Reihenfolge die $\tau$ Zeitintervalle.
In einem solchen $\benbrace*{t_u,t_{u+1}}$ bewegen sich $x$ und $y$ nur entlang je einer Geraden, sodass der Abstand eine konvexe Funktion beschreibt.
Insbesondere gibt es entweder $k$ mit $k \in \set*{0,1,2}$ Zeitpunkte aus $\benbrace*{t_u,t_{u+1}}$, an denen der Abstand genau $2 \varepsilon$ oder dies gilt für jeden Zeitpunkt $t \in \benbrace*{t_u,t_{u+1}}$.
Es lässt sich in $\mathcal{O}(1)$ Zeit bestimmen, welcher der Fälle eintritt, und welches ggf. die $k$ Zeitpunkte sind.
Im ersten Fall fügen wir die $k$ Zeitpunkte in die Liste ein, im zweiten nur die Randpunkte $t_u$ und $t_{u+1}$; in beiden Fällen fügen wir die Zeitpunkte nicht ein, falls dadurch die Eindeutigkeit verletzt werden würde (asymptotisch kein zusätzlicher Aufwand, da nur das zuletzt eingefügte Element betrachtet werden muss).

In einem zweiten Schritt bereinigen wir $L_{xy}$ nun wie folgt: Wir iterieren über $L_{xy}$ und berechnen dabei für jeden Eintrag $t_i$, ob $x$ und $y$ im offenen Intervall $(t_i,t_{i+1})$ direkt-zusammenhängend sind (Abstand beim Mittelpunkt bestimmen). 
Finden wir dadurch Zeitpunkte in $L_{xy}$, an denen sich der Zusammenhang von $x$ und $y$ nicht ändert, so entfernen wir diese aus der $L_{x,y}$.\todo{Zeichnung, dass der Fall tatsächlich eintreten kann}
Die verbleibenden beschriften wir entsprechend mit \emph{Connect} oder \emph{Disconnect}.

$L_{xy}$ besteht aus $\mathcal{O}(\tau)$ Zeitpunkten und alle obigen Berechnungen kosten zusammen $\Theta(\tau)$ Zeit.
Berechnen wir diese Listen für alle $x,y$ und Vereinigen sie mittels \textsc{MergeSort} zu einer Liste $L$, so entstehen also insgesamt $\mathcal{O}(\tau n^2 \log n^2) = \mathcal{O}(\tau n^2 \log n)$ Kosten und $L$ enthält $\mathcal{O}(\tau n^2)$ Zeitpunkte.
% paragraph ereignisse_berechnen (end)


% section berechnung_reeb (end)