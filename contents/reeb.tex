%!TEX root = ../trajectory-grouping.tex
Sei $X$ ein topologischer Raum und $f \colon X \to \mathbb{R}$ eine stetige Funktion.
Dann bezeichnet man für einen Wert $a \in \mathbb{R}$ das Urbild unter $f$ als \Index{Niveaumenge} (engl. \emph{level set}) von $a$.
Die Niveaumengen bilden offensichtlich eine Partition von $X$.

Anschaulich stellt man sich $f$ meist als \Index{Höhenfunktion} vor. 
So könnte $X$ beispielsweise das Gebiet einer Karte in einem handelsüblichen Atlas sein (=eine Teilmenge der zweidimensionalen Ebene) und $f$ gibt die Höhe über NN an.
Dann sind die Niveaumengen $f^{-1}(\set{a})$ genau die Niveaulinien, die man in detaillierten topografischen Karten findet (ein Beispiel einer solchen Karte findet sich in \cref{sec:karte_niveau}).

Um den \Index{Reeb-Graph} von $f$ zu erhalten, benutzen wir nun die Zusammenhangskomponenten von $f^{-1}(\set{a})$ wie folgt (Definition nach \textcite[\RM{6}.4]{compTopo})

\begin{definition}[{name=[Reeb-Graph]}]
	Der \Index{Reeb-Graph} zu $f \colon X \to \mathbb{R}$ ist der Quotientenraum $\mathcal{R}(f) \coloneqq X/{\sim}$ bezüglich der folgenden Äquivalenzrelation
	\[
		x \sim y \,:\mkern-3mu\iff \exists a \in \mathbb{R} : x,y \text{ liegen in der gleichen Zusammenhangskomponente von } f^{-1}(\set{a}) 
	\]
	Die Elemente von $\mathcal{R}(f)$ bezeichnen wir als \Index{Konturen} von $f$.
\end{definition}
Wir erhalten eine eindeutig bestimme Abbildung $h \colon \mathcal{R}(f) \to \mathbb{R}$, sodass das folgende Diagramm kommutiert.
Alle Abbildungen sind aufgrund der universellen Eigenschaft der Quotiententopologie stetig.
\[
	\begin{tikzcd}[sep=large]
		X \rar["f"] \dar["q",two heads] & \mathbb{R} \\
		\mathcal{R}(f) \urar["h"']
	\end{tikzcd}
\]
Die Quotientenabbildung $q$ bildet einen Punkt aus $X$ auf die zugehörige Kontur in $\mathcal{R}(f)$ ab.
$\mathcal{R}(f)$ ist tatsächlich ein Graph, da jede Niveaumenge unter $q$ auf eine diskrete Menge abgebildet wird.\todo{genauer bzw. expliziter}

\section{Differentialtopologische Hintergründe zum Reeb-Graphen} % (fold)
\label{sec:background_reeb}
\begin{figure}[tbhp]
	\Centering
	\begin{tikzpicture}[rotate=90, xscale=1, yscale=1, scale=0.7]
		% \draw[help lines] (-4,-10) grid (4,4);
		% reelle Gerade
		\draw[->,thick] (-4,-6) -- ++ (8,0) node[left=3pt]{$\mathbb{R}$};
		
		\draw[DodgerBlue3, thick] (-2.5,1.89) .. controls +(240:0.8) and +(120:0.8) .. (-2.5,-1.89);
		\draw[DodgerBlue3, thick, dashed] (-2.5,1.89) .. controls +(300:0.8) and +(60:0.8) .. (-2.5,-1.89);
		\draw[fill=DodgerBlue3] (-2.5,-6) circle[radius=0.07];
		
		\draw[Firebrick2, thick] (0,2.5) .. controls +(250:0.6) and +(110:0.6) .. (0,0.5);
		\draw[Firebrick2, thick, dashed] (0,2.5) .. controls +(290:0.6) and +(70:0.6) .. (0,0.5);
		
		\draw[Firebrick2, thick] (0,-0.5) .. controls +(250:0.6) and +(110:0.6) .. (0,-2.5);
		\draw[Firebrick2, thick, dashed] (0,-0.5) .. controls +(290:0.6) and +(70:0.6) .. (0,-2.5);
		\draw[fill=Firebrick2] (0,-6) circle[radius=0.07];
		
		\draw[SeaGreen3, thick] (1.75,2.25) .. controls +(250:0.6) and +(110:0.6) .. (1.75,0) .. controls +(250:0.6) and +(110:0.6) .. (1.75,-2.25);
		\draw[SeaGreen3, thick, dashed] (1.75,2.25) .. controls +(290:0.6) and +(70:0.6) .. (1.75,0) .. controls +(290:0.6) and +(70:0.6) .. (1.75,-2.25);
		\draw[fill=SeaGreen3] (1.75,-6) circle[radius=0.07];
		
		\draw[SeaGreen3, thick] (-1.75,2.25) .. controls +(250:0.6) and +(110:0.6) .. (-1.75,0) .. controls +(250:0.6) and +(110:0.6) .. (-1.75,-2.25);
		\draw[SeaGreen3, thick, dashed] (-1.75,2.25) .. controls +(290:0.6) and +(70:0.6) .. (-1.75,0) .. controls +(290:0.6) and +(70:0.6) .. (-1.75,-2.25);
		\draw[fill=SeaGreen3] (-1.75,-6) circle[radius=0.07];
		
		% kritische Werte and Start und Ende
		\begin{scope}
			\clip (-3.5,-1) rectangle (3.5,1);
			\draw[Sienna2,fill=Sienna2] (3.5,0) circle[radius=.08];
			\draw[Sienna2,fill=Sienna2] (-3.5,0) circle[radius=.08];
		\end{scope}
		\draw[fill=Sienna2] (3.5,-6) circle[radius=.07];
		\draw[fill=Sienna2] (-3.5,-6) circle[radius=.07];
		
		% draw reeb graph
		\begin{scope}[every node/.style={shape=circle,inner sep=1.7pt,fill}]
			\draw[thick] (-3.5,-12) node{} -- ++(1.75,0) .. controls +(80:1.5) and +(100:1.5) .. ++(3.5,0) node{};
			\draw[thick] (3.5,-12) node{}-- ++(-1.75,0) .. controls +(-100:1.5) and +(-80:1.5) .. ++(-3.5,0) node{};
		\end{scope}
		\node at (3,-11) {$\mathcal{R}(f)$}; 
		
		\draw[thick] (-3.5,0) .. controls (-3.5,2) and (-1.5,2.5) .. (0,2.5);
		\draw[thick,xscale=-1] (-3.5,0) .. controls (-3.5,2) and (-1.5,2.5) .. (0,2.5);
		\draw[thick,rotate=180] (-3.5,0) .. controls (-3.5,2) and (-1.5,2.5) .. (0,2.5);
		\draw[thick,yscale=-1] (-3.5,0) .. controls (-3.5,2) and (-1.5,2.5) .. (0,2.5);
		\draw[thick] (-2,.2) .. controls (-1.5,-0.3) and (-1,-0.5) .. (0,-.5) .. controls (1,-0.5) and (1.5,-0.3) .. (2,0.2);
		\draw[thick] (-1.75,0) .. controls (-1.5,0.3) and (-1,0.5) .. (0,.5) .. controls (1,0.5) and (1.5,0.3) .. (1.75,0);
		
		\draw[-to] (0,-3.5) -- (0,-5) node[above, midway]{$f$};
	\end{tikzpicture}
	\caption[Höhenfunktion des aufrechten Torus mit dem dazugehörigen Reeb-Graphen]{Höhenfunktion des aufrechten Torus $T^2=S^1 \times S^1$ mit dem dazugehörigen Reeb-Graphen. Die grün und orange gekennzeichneten Niveaumengen enthalten je einen kritischen Punkt, die rote und blaue enthalten ausschließlich reguläre Werte. $f$ ist eine Morse-Funktion und die Knoten des Reeb-Graphen stehen in Bijektion zu den kritischen Punkten.}\label{fig:torus_reeb}
\end{figure}
Obwohl wir den Reeb-Graph allgemein definiert haben und auch im weiteren Verlauf keine differenzierbare Struktur auf $X$ benötigen, ist es an dieser Stelle sinnvoll einen kurzen Ausflug in die Differentialtopologie zu machen, um mit einem instruktiven Beispiel zu erläutern, wo genau die Knoten im Reeb-Graphen \enquote{herkommen}.
Neben den gleich auch direkt referenzierten Büchern von \textcite{compTopo} und \textcite{MilnorMorse}, sei an dieser Stelle auf das Buch \citetitle{Miln} von \textcite{Miln} verwiesen, welches einen schnellen, lesenswerten Einstieg in die Grundlagen der Differentialtopologie vermittelt.

Betrachtet man nicht einen beliebigen topologischen Raum $X$, sondern eine differenzierbare Mannigfaltigkeit $M$ wie zum Beispiel den 2-Torus (siehe \cref{fig:torus_reeb}), und eine glatte Funktion $f \colon M \to \mathbb{R}$ so kann man die Punkte von $M$ anhand des Differentials von $f$ in dem jeweiligen Punkt in sogenannte \bet{kritische}\index{kritische Punkte} und \Index{reguläre Punkte} unterteilen.
Für den aufrechten Torus und die Höhenfunktion sind dies die beiden orange gekennzeichneten, einelementigen Niveaumengen in \cref{fig:torus_reeb}, sowie die beiden \enquote{Klebepunkte} der grün gezeichneten Niveaumengen.

Durch das Betrachten der zweiten Ableitungen in lokalen Koordinaten, kann man die kritischen Punkte anhand der Hesse-Matrix in entartete und nicht entartete einteilen. 
Nach \textcite{compTopo} heißt $f$ dann \Index{Morse-Funktion}, wenn 
\begin{enumerate}[(i)]
	\item alle kritischen Punkte nicht entartet sind und\marginnote{einige Autoren verzichten auf \cref{enum:def:morse:2}}
	\item\label{enum:def:morse:2} die kritischen Punkte paarweise verschiedene Werte haben. 
\end{enumerate}
Beliebige glatte Funktionen $f \colon M \to \mathbb{R}$ lassen sich nach \textcite[Corr.~6.8]{MilnorMorse} gut durch Morse-Funktionen approximieren.

Setze $M^a \coloneqq f^{-1}\enbrace*{(-\infty,a]}$.
Die zentrale Erkenntnis der Morse-Theorie ist es nun, dass sich die Topologie von $M^a$ bei Variation von $a$ nur an den kritischen Werten, also den Bildern kritischer Punkte, ändert.
Darüber hinaus kann diese Änderung als das Ankleben von Zellen realisiert werden.

Für den aufrechten Torus $T$ aus \cref{fig:torus_reeb} ist $T^a$ zum Beispiel zunächst topologisch gesehen eine Scheibe, wird beim Passieren der zweiten kritischen Punktes zu einem Zylinder, beim dritten zu einem Torus, aus dem eine Scheibe ausgeschnitten wurde, und schlussendlich am Maximum zum eigentlichen Torus.

In Bezug auf den Reeb-Graphen bedeutet dies, dass sich auch die Zusammenhangskomponenten der Niveaumengen nun an den kritischen Werten ändern können.
Ein Punkt $r \in \mathcal{R}(f)$ ist also ein Knoten, wenn $q^{-1}(u)$ einen kritischen Punkt enthält.
\Cref{enum:def:morse:2} stellt dann sicher, dass es eine Bijektion zwischen den Knoten im Reeb-Graphen und den kritischen Punkten gibt, die in \cref{fig:torus_reeb} klar ersichtlich ist.

% section background_reeb (end)



\section{Reeb-Graphen zur Gruppierung von Trajektorien} % (fold)
\label{sec:trajek_reeb}
Sei $\mathcal{X}$ wieder eine Menge von Entitäten, die sich entlang bekannter Trajektorien bewegen, die jeweils aus $\tau$ Kanten bestehen.
Indem wir die $z$-Achse des $\mathbb{R}^3$ als Zeit auffassen erhalten wir für jede Entität $x$ durch das Verfolgen\todo{besseres Wort?} der $\varepsilon$-Scheibe entlang eines Segments der Trajektorie einen schiefen Zylinder im $\mathbb{R}^3$.
Die Vereinigung der $\tau$ Zylinder liefert uns dann einen (ausgefüllten) \enquote{Schlauch} pro Entität, welche wir wiederum alle zu einem Teilraum\todo{Mfkt?} $\mathcal{M} \subset \mathbb{R}^3$ vereinigen (siehe \cref{fig:tubes}).
Dies ist der Raum, dessen Reeb-Graph wir betrachten wollen.

\begin{figure}[htbp]
	\Centering
	\missingfigure{Diese lustigen Tuben}
	\caption{Der Raum $\mathcal{M}$ für die Entitäten $x_1, \ldots ,x_k$}\label{fig:tubes}
\end{figure}

% section trajek_reeb (end)

