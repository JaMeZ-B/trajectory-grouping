%!TEX root = ../trajectory-grouping.tex
Sei $X$ ein topologischer Raum und $f \colon X \to \mathbb{R}$ eine stetige Funktion.
Dann bezeichnet man für einen Wert $a \in \mathbb{R}$ das Urbild unter $f$ als \Index{Niveaumenge} (engl. \emph{level set}) von $a$.
Die Niveaumengen bilden offensichtlich eine Partition von $X$.

Anschaulich stellt man sich $f$ meist als \Index{Höhenfunktion} vor. 
So könnte $X$ beispielsweise das Gebiet einer Karte in einem handelsüblichen Atlas sein (=eine Teilmenge der zweidimensionalen Ebene) und $f$ gibt die Höhe über NN an.
Dann sind die Niveaumengen $f^{-1}(\set{a})$ genau die Niveaulinien, die man in detaillierten topografischen Karten findet (ein Beispiel einer solchen Karte findet sich in \cref{sec:karte_niveau}).

Um den \Index{Reeb-Graph} von $f$ zu erhalten, benutzen wir nun die Zusammenhangskomponenten von $f^{-1}(\set{a})$ wie folgt (Definition nach \textcite[\RM{6}.4]{compTopo})

\begin{definition}[{name=[Reeb-Graph]}]
	Der \Index{Reeb-Graph} zu $f \colon X \to \mathbb{R}$ ist der Quotientenraum $\mathcal{R}(f) \coloneqq X/{\sim}$ bezüglich der folgenden Äquivalenzrelation
	\[
		x \sim y \,:\mkern-3mu\iff \exists a \in \mathbb{R} : x,y \text{ liegen in der gleichen Zusammenhangskomponente von } f^{-1}(\set{a}) 
	\]
	Die Elemente von $\mathcal{R}(f)$ bezeichnen wir als \Index{Konturen} von $f$.
\end{definition}
Wir erhalten eine eindeutig bestimme Abbildung $h \colon \mathcal{R}(f) \to \mathbb{R}$, sodass das folgende Diagramm kommutiert.
Alle Abbildungen sind aufgrund der universellen Eigenschaft der Quotiententopologie stetig.
\[
	\begin{tikzcd}[sep=large]
		X \rar["f"] \dar["q",two heads] & \mathbb{R} \\
		\mathcal{R}(f) \urar["h"']
	\end{tikzcd}
\]
Die Quotientenabbildung $q$ bildet einen Punkt aus $X$ auf die zugehörige Kontur in $\mathcal{R}(f)$ ab.
$\mathcal{R}(f)$ ist tatsächlich ein Graph, da jede Niveaumenge unter $q$ auf eine diskrete Menge abgebildet wird.\todo{genauer bzw. expliziter}
Betrachtet man die positive Richtung in $\mathbb{R}$, so wird $\mathcal{R}(f)$ auf kanonische Weise zu einem gerichteten Graphen.

\section{Differentialtopologische Hintergründe zum Reeb-Graphen} % (fold)
\label{sec:background_reeb}
\begin{figure}[tbhp]
	\Centering
	\begin{tikzpicture}[rotate=90, xscale=1, yscale=1, scale=0.7]
		% \draw[help lines] (-4,-10) grid (4,4);
		% reelle Gerade
		\draw[->,thick] (-4,-6) -- ++ (8,0) node[left=3pt]{$\mathbb{R}$};
		
		\draw[DodgerBlue3, thick] (-2.5,1.89) .. controls +(240:0.8) and +(120:0.8) .. (-2.5,-1.89);
		\draw[DodgerBlue3, thick, dashed] (-2.5,1.89) .. controls +(300:0.8) and +(60:0.8) .. (-2.5,-1.89);
		\draw[fill=DodgerBlue3] (-2.5,-6) circle[radius=0.07];
		
		\draw[Firebrick2, thick] (0,2.5) .. controls +(250:0.6) and +(110:0.6) .. (0,0.5);
		\draw[Firebrick2, thick, dashed] (0,2.5) .. controls +(290:0.6) and +(70:0.6) .. (0,0.5);
		
		\draw[Firebrick2, thick] (0,-0.5) .. controls +(250:0.6) and +(110:0.6) .. (0,-2.5);
		\draw[Firebrick2, thick, dashed] (0,-0.5) .. controls +(290:0.6) and +(70:0.6) .. (0,-2.5);
		\draw[fill=Firebrick2] (0,-6) circle[radius=0.07];
		
		\draw[SeaGreen3, thick] (1.75,2.25) .. controls +(250:0.6) and +(110:0.6) .. (1.75,0) .. controls +(250:0.6) and +(110:0.6) .. (1.75,-2.25);
		\draw[SeaGreen3, thick, dashed] (1.75,2.25) .. controls +(290:0.6) and +(70:0.6) .. (1.75,0) .. controls +(290:0.6) and +(70:0.6) .. (1.75,-2.25);
		\draw[fill=SeaGreen3] (1.75,-6) circle[radius=0.07];
		
		\draw[SeaGreen3, thick] (-1.75,2.25) .. controls +(250:0.6) and +(110:0.6) .. (-1.75,0) .. controls +(250:0.6) and +(110:0.6) .. (-1.75,-2.25);
		\draw[SeaGreen3, thick, dashed] (-1.75,2.25) .. controls +(290:0.6) and +(70:0.6) .. (-1.75,0) .. controls +(290:0.6) and +(70:0.6) .. (-1.75,-2.25);
		\draw[fill=SeaGreen3] (-1.75,-6) circle[radius=0.07];
		
		% kritische Werte and Start und Ende
		\begin{scope}
			\clip (-3.5,-1) rectangle (3.5,1);
			\draw[Sienna2,fill=Sienna2] (3.5,0) circle[radius=.08];
			\draw[Sienna2,fill=Sienna2] (-3.5,0) circle[radius=.08];
		\end{scope}
		\draw[fill=Sienna2] (3.5,-6) circle[radius=.07];
		\draw[fill=Sienna2] (-3.5,-6) circle[radius=.07];
		
		% draw reeb graph
		\begin{scope}[every node/.style={shape=circle,inner sep=1.7pt,fill}]
			\draw[thick] (-3.5,-12) node{} -- ++(1.75,0) .. controls +(80:1.5) and +(100:1.5) .. ++(3.5,0) node{};
			\draw[thick] (3.5,-12) node{}-- ++(-1.75,0) .. controls +(-100:1.5) and +(-80:1.5) .. ++(-3.5,0) node{};
		\end{scope}
		\node at (3,-11) {$\mathcal{R}(f)$}; 
		
		\draw[thick] (-3.5,0) .. controls (-3.5,2) and (-1.5,2.5) .. (0,2.5);
		\draw[thick,xscale=-1] (-3.5,0) .. controls (-3.5,2) and (-1.5,2.5) .. (0,2.5);
		\draw[thick,rotate=180] (-3.5,0) .. controls (-3.5,2) and (-1.5,2.5) .. (0,2.5);
		\draw[thick,yscale=-1] (-3.5,0) .. controls (-3.5,2) and (-1.5,2.5) .. (0,2.5);
		\draw[thick] (-2,.2) .. controls (-1.5,-0.3) and (-1,-0.5) .. (0,-.5) .. controls (1,-0.5) and (1.5,-0.3) .. (2,0.2);
		\draw[thick] (-1.75,0) .. controls (-1.5,0.3) and (-1,0.5) .. (0,.5) .. controls (1,0.5) and (1.5,0.3) .. (1.75,0);
		
		\draw[-to] (0,-3.5) -- (0,-5) node[above, midway]{$f$};
	\end{tikzpicture}
	\caption[Höhenfunktion des aufrechten Torus mit dem dazugehörigen Reeb-Graphen]{Höhenfunktion des aufrechten Torus $T^2=S^1 \times S^1$ mit dem dazugehörigen Reeb-Graphen. Die grün und orange gekennzeichneten Niveaumengen enthalten je einen kritischen Punkt, die rote und blaue enthalten ausschließlich reguläre Werte. $f$ ist eine Morse-Funktion und die Knoten des Reeb-Graphen stehen in Bijektion zu den kritischen Punkten.}\label{fig:torus_reeb}
\end{figure}
Obwohl wir den Reeb-Graph allgemein definiert haben und auch im weiteren Verlauf keine differenzierbare Struktur auf $X$ benötigen, ist es an dieser Stelle sinnvoll einen kurzen Ausflug in die Differentialtopologie zu machen, um mit einem instruktiven Beispiel zu erläutern, wo genau die Knoten im Reeb-Graphen \enquote{herkommen}.
Neben den gleich auch direkt referenzierten Büchern von \textcite{compTopo} und \textcite{MilnorMorse}, sei an dieser Stelle auf das Buch \citetitle{Miln} von \textcite{Miln} verwiesen, welches einen schnellen, lesenswerten Einstieg in die Grundlagen der Differentialtopologie vermittelt.

Betrachtet man nicht einen beliebigen topologischen Raum $X$, sondern eine differenzierbare Mannigfaltigkeit $M$ wie zum Beispiel den 2-Torus (siehe \cref{fig:torus_reeb}), und eine glatte Funktion $f \colon M \to \mathbb{R}$ so kann man die Punkte von $M$ anhand des Differentials von $f$ in dem jeweiligen Punkt in sogenannte \bet{kritische}\index{kritische Punkte} und \Index{reguläre Punkte} unterteilen.
Für den aufrechten Torus und die Höhenfunktion sind dies die beiden orange gekennzeichneten, einelementigen Niveaumengen in \cref{fig:torus_reeb}, sowie die beiden \enquote{Klebepunkte} der grün gezeichneten Niveaumengen.

Durch das Betrachten der zweiten Ableitungen in lokalen Koordinaten, kann man die kritischen Punkte anhand der Hesse-Matrix in entartete und nicht entartete einteilen. 
Nach \textcite{compTopo} heißt $f$ dann \Index{Morse-Funktion}, wenn 
\begin{enumerate}[(i)]
	\item alle kritischen Punkte nicht entartet sind und\marginnote{einige Autoren verzichten auf \cref{enum:def:morse:2}}
	\item\label{enum:def:morse:2} die kritischen Punkte paarweise verschiedene Werte haben. 
\end{enumerate}
Beliebige glatte Funktionen $f \colon M \to \mathbb{R}$ lassen sich nach \textcite[Corr.~6.8]{MilnorMorse} gut durch Morse-Funktionen approximieren.

Setze $M^a \coloneqq f^{-1}\enbrace*{(-\infty,a]}$.
Die zentrale Erkenntnis der Morse-Theorie ist es nun, dass sich die Topologie von $M^a$ bei Variation von $a$ nur an den kritischen Werten, also den Bildern kritischer Punkte, ändert.
Darüber hinaus kann diese Änderung als das Ankleben von Zellen realisiert werden.

Für den aufrechten Torus $T$ aus \cref{fig:torus_reeb} ist $T^a$ zum Beispiel zunächst topologisch gesehen eine Scheibe, wird beim Passieren der zweiten kritischen Punktes zu einem Zylinder, beim dritten zu einem Torus, aus dem eine Scheibe ausgeschnitten wurde, und schlussendlich am Maximum zum eigentlichen Torus.

In Bezug auf den Reeb-Graphen bedeutet dies, dass sich auch die Zusammenhangskomponenten der Niveaumengen nun an den kritischen Werten ändern können.
Ein Punkt $r \in \mathcal{R}(f)$ ist also ein Knoten, wenn $q^{-1}(u)$ einen kritischen Punkt enthält.
\Cref{enum:def:morse:2} stellt dann sicher, dass es eine Bijektion zwischen den Knoten im Reeb-Graphen und den kritischen Punkten gibt, die in \cref{fig:torus_reeb} klar ersichtlich ist.
% section background_reeb (end)



\section{Reeb-Graphen zur Gruppierung von Trajektorien} % (fold)
\label{sec:trajek_reeb}
Sei $\mathcal{X}$ wieder eine Menge von Entitäten, die sich entlang bekannter Trajektorien bewegen, die jeweils aus $\tau$ Kanten bestehen.
Indem wir die $z$-Achse des $\mathbb{R}^3$ als Zeit auffassen erhalten wir für jede Entität $x$ durch Extrusion der $\varepsilon$-Scheibe entlang eines Segments der Trajektorie einen schiefen Zylinder im $\mathbb{R}^3$.
Die Vereinigung der $\tau$ Zylinder liefert uns dann einen (ausgefüllten) \enquote{Schlauch} pro Entität, welche wir wiederum alle zu einem Teilraum\todo{Mfkt?} $\mathcal{M} \subset \mathbb{R}^3$ vereinigen (siehe \cref{fig:tubes}).
Dies ist der Raum, dessen Reeb-Graph wir betrachten wollen.

\begin{figure}[htbp]
	\Centering
	
	\subfloat[Der Raum $\mathcal{M}$]{
	\missingfigure[figwidth=.45\textwidth]{Diese lustigen Tuben }\label{fig:tubes}
	}
	\hspace{1em}
	\subfloat[Reeb-Graph zu $\mathcal{M}$]{
	\missingfigure[figwidth=.45\textwidth]{Reeb-Graph}}\label{fig:reeb_graph}
	\caption{Der Raum $\mathcal{M}$ für die Entitäten $x_1, \ldots ,x_k$}
\end{figure}

Bezeichnet man nun mit $H_t$ die horizontale Ebene bei der Höhe $t$, so sind die Mengen $\mathcal{M} \cap H_t$ offensichtlich die Niveaumengen.
Eine Zusammenhangskomponente von $\mathcal{M} \cap H_t$ entspricht genau einer Komponente im Sinne von \cref{cha:def_gruppe}, also einer maximalen Menge $\varepsilon$-zusammenhängender Entitäten zum Zeitpunkt $t$.
Der Einfachheit halber nehmen wir im weiteren Verlauf folgendes an:
\begin{itemize}
	\item Die bekannten Positionen der Trajektorien sind alle an gleichen Zeitpunkten $t_0, \ldots ,t_\tau$.
	\item Keine drei Entitäten werden zur gleichen Zeit $\varepsilon$-(un)zusammenhängend.
\end{itemize}
Es wird sich zeigen, dass diese Annahmen ohne Beschränkung der Allgemeinheit getroffen werden können.
Die erste Annahme lässt sich durch das Zerteilen einzelner Segmente der Trajektorien zu geeigneten Zeitpunkten auch leicht künstlich herstellen.

\begin{definition}[{name=[Reeb-Graph der \GrpStruktur]}]
	Wir bezeichnen mit $\mathcal{R}$ den Reeb-Graphen zu $\mathcal{M}$ bezüglich der Höhenfunktion zu $\mathcal{M}$.
\end{definition}

An dieser Stelle sei angemerkt, dass $\mathcal{M}$ ein dreidimensionaler Raum ist (bis auf einige wenige Ausnahmen ist $\mathcal{M}$ eine 3-Mannigfaltigkeit mit Rand; in jedem Fall ist $\mathcal{M}$ ein dreidimensionaler CW-Komplex) im Gegensatz zum zweidimensionalen Torus, der eben betrachtet wurde.
Das in \cref{fig:bsp_reeb_rand} betrachtete Beispiel zeigt, warum wir den dreidimensionalen Raum $\mathcal{M}$ und nicht dessen zweidimensionalen Rand betrachten.

\begin{figure}[htbp]
	\Centering
	\begin{tikzpicture}[scale=1.3]
		\foreach \x in {0,30,...,330}{
			\filldraw[Firebrick2] (\x:1) circle[radius=.4];
		}
		\foreach \x in {0,30,...,330}{
			\fill[white] (\x:1) circle[radius=.37];
		}
		\foreach \x in {0,30,...,330}{
			\fill[DodgerBlue3,fill opacity=.5] (\x:1) circle[radius=.37];
			\filldraw (\x:1) circle[radius=.04];
		}
 	\end{tikzpicture}
	\caption[Unterschiede bei den Zusammenhangskomponenten zwischen $\mathcal{M}$ und $\partial \mathcal{M}$]{Angenommen die gezeichneten Entitäten bewegen sich gar nicht. Jede Niveaumenge von $\mathcal{M}$ ist dann zusammenhängend, die Niveaumengen von $\partial \mathcal{M}$ bestehen aber stets aus zwei Zusammenhangskomponenten. Da obige Konfiguration (bei geeignetem $\delta$) nur eine maximale Gruppe bildet, wurde $\mathcal{R}$ als Reeb-Graph von $\mathcal{M}$ und nicht $\partial \mathcal{M}$ definiert.}\label{fig:bsp_reeb_rand}
\end{figure}

Wie in \cref{sec:background_reeb} erläutert, entsprechen die Vertices von $\mathcal{R}$ Zeitpunkten $t_v$, an denen sich die Komponenten ändern.
Diese Zeitpunkte stimmen im Allgemeinen nicht mit den Zeitpunkten $t_0, \ldots, t_\tau$ überein, was auch in \cref{fig:tubes} zu erkennen ist.
Jeder Vertex im Reeb-Graphen lässt sich eindeutig einer der folgenden Klassen zuordnen
\begin{description}
	\item[Start-Vertex] Für jede Komponente zum Zeitpunkt $t_0$ existiert ein Vertex; ein solcher Vertex ist ein \Index{Start-Vertex}.
	Jeder Start-Vertex hat Eingangs-Grad 0 und Ausgangsgrad 1.
	\item[End-Vertex] Genauso existiert für jede Komponente zum Zeitpunkt $t_\tau$ ein Vertex, ein sog. \Index{End-Vertex}.
	Jeder End-Vertex hat Eingangs-Grad 1 und Ausgangsgrad 0.
\end{description}
Die verbleibenden Vertices können wir aufgrund der Annahme, dass keine drei Entitäten zur gleichen Zeit $\varepsilon$-(un)zusammenhängend werden, eindeutig einer der beiden folgenden Klassen zuordnen: Ein
\begin{description}
	\item[Merge-Vertex] hat Eingangs-Grad 2 und Ausgangsgrad 1, ein
	\item[Split-Vertex] hat Eingangs-Grad 1 und Ausgangsgrad 2.
\end{description}
Eine gerichtete Kante $e=(u,v)$ in $\mathcal{R}$ entspricht einer Menge von Entitäten, die zu jedem Zeitpunkt $t \in \benbrace*{t_u,t_v}$ eine Komponente (im Sinne von \cref{cha:def_gruppe}) bildet.
Wir halten fest, dass der Reeb-Graph $\mathcal{R}$ nur vom räumlichen Parameter $\varepsilon$, aber nicht von den beiden weiteren Parametern unseres Modells, $m$ und $\delta$, abhängt.

Wie genau hängt der Reeb-Graph $\mathcal{R}$ nun mit den Trajektorien und den Gruppen zusammen? 
Zunächst einmal folgt jede Entität einem Pfad in $\mathcal{R}$ von einem Start-Vertex zu einem End-Vertex.
Analog lässt sich jeder (maximalen) Gruppe ein Pfad von einem Start- oder Merge-Vertex zu einem End- oder Split-Vertex zuordnen.
Wählt man nun $m > 1$ oder $\delta > 0$, so kann es sein, dass einigen Kanten von $\mathcal{R}$ keine Gruppe zugeordnet ist und sie somit für die \GrpStruktur irrelevant sind.
$\mathcal{R}$ ohne all diese Kanten ist der \Index{reduzierte Reeb-Graph}, den wir ebenfalls mit $\mathcal{R}$ bezeichnen (nur für die Komplexität von $\mathcal{R}$ nehmen wir gleich noch einmal den unreduzierten Fall an).


\begin{definition}[{name=[{\GrpStruktur}]},label=def:grpstrukur]
	Die \Index{\GrpStruktur} (\enquote{trajectory grouping structure}) zu $\mathcal{X}$ und den Parametern $(\varepsilon,\delta,m)$ besteht aus dem reduzierten Reeb-Graphen $\mathcal{R}$ und einer Abbildung, die jeder Kante die zugehörige Menge maximaler Gruppen zuordnet.
\end{definition}
% section trajek_reeb (end)

\section{Komplexität der \GrpStruktur} % (fold)
\label{sec:komplex}
Gemäß \cref{def:grpstrukur} müssen wir sowohl für die Komplexität des Reeb-Graphen $\mathcal{R}$ als auch die Anzahl maximaler Gruppen obere Schranken finden.
Dazu folgen wir \textcite[Sec.~2.1~\&~2.2]{buchin2015} und beweisen wir Schranken von $\mathcal{O}(\tau n^2)$ für $\mathcal{R}$ und $\mathcal{O}(\tau n^3)$ für die Anzahl maximaler Gruppen, die im Worst-Case sogar scharf sind. 

\begin{lemma}[name={\cite[Lem.~1]{buchin2015}},label=lem:untereSchranke-Reeb]
	Der Reeb-Graph $\mathcal{R}$ einer Menge $\mathcal{X}$ von $n$ Entitäten, von denen sich jede auf einer Trajektorie mit $\tau$ Kanten bewegt, kann $\Omega(\tau n^2)$ Vertices und $\Omega(\tau n^2)$ Kanten haben.
\end{lemma}
\begin{beweis}
	Wir nehmen ohne Einschränkungen an, dass $n$ gerade sei; weiter können wir $\varepsilon=0$ wählen.
	Wir wollen nun eine Menge von Trajektorien konstruieren, sodass $\mathcal{R}$ $\Omega(n^2)$ viele Vertices $v$ mit $t_v \in \benbrace*{t_i,t_{i+1}}$ hat für einen der Zeitpunkte $t_i$.
	Diese Konstruktion wiederholen wir für alle Intervalle.
	Dies liefert dann offensichtlich $\Omega(\tau n^2)$ viele Vertices und damit auch $\Omega(\tau n^2)$ viele Kanten, da jeder Knoten entweder Grad 1 oder Grad 3 hat (siehe oben).

	\begin{figure}[htbp]
		\Centering
		\begin{tikzpicture}[scale=.3,>=Latex,font=\small]
			\draw[dashed] (-7,-7) node[left]{$t_i$} -- (7,7);
			\draw[dashed] (0,-12) node[left]{$t_{i+1}$} -- ++ (14,14);
			\foreach \x in {1,2,3,4,5}{
				\draw[->,thick] (-\x,-\x) node[left]{$r_{\x}$} -- ++ (12,0);
				\draw[->,thick] (\x,\x) node[above,xshift=-1pt]{$d_{\x}$} -- ++ (0,-12);
				\filldraw[fill=DodgerBlue3] (-\x,-\x) circle[radius=.15] ++ (12,0)circle[radius=.15];
				\filldraw[fill=Firebrick2] (\x,\x) circle[radius=.15] ++ (0,-12) circle[radius=.15];
			}
		\end{tikzpicture}
		\caption{Konstruktion für $\Omega(n^2)$ viele Vertices im Intervall $\benbrace*{t_i,t_{i+1}}$ am Beispiel $n=5$}\label{fig:vertices-intervall}
	\end{figure}
	
	Wir teilen $\mathcal{X}$ -- wie in \cref{fig:vertices-intervall} gezeichnet -- in zwei Teilmengen $R= \set*{r_1, \ldots ,r_{n/2}}$, $D=\set*{d_1,\ldots ,d_{n/2}}$ auf, wobei die Entität $r_j$ bei $(-j,-j)$ und $d_\ell$ bei $(\ell,\ell)$ startet.
	Alle Entitäten bewegen sich mit der Geschwindigkeit Eins, die $r_j$ nach rechts, die $d_\ell$ nach unten, und aus \cref{fig:vertices-intervall} ist klar ersichtlich, dass sich die Konstruktion mit vertauschten Rollen im nächsten Intervall wiederholen lässt.
	Da sich $r_j$ und $d_\ell$ zum Zeitpunkt $t_i + j + \ell$ treffen, entsteht ein Vertex im Reeb-Graphen und wir erhalten insgesamt $\Omega(n^2)$ Vertices pro Zeitintervall.
\end{beweis}

\begin{satz}[name={\cite[Thm.~2]{buchin2015}},label=thm:schranke-Reeb]
	Der Reeb-Graph $\mathcal{R}=(V,E)$ aus \cref{lem:untereSchranke-Reeb} hat $\mathcal{O}(\tau n^2)$ Vertices und Kanten.
	Diese Schranke ist scharf im Worst-Case.
\end{satz}
\begin{beweis}
	In \cref{lem:untereSchranke-Reeb} wurde bereits eine Konstruktion angegeben, die einen Reeb-Graphen mit $\Omega(\tau n^2)$ Vertices und Kanten liefert.
	Wir müssen also nur noch zeigen, dass $\mathcal{O}(\tau n^2)$ Vertices und Kanten entstehen können.
	Dazu nutzen wir aus, dass sich die Entitäten während eines Zeitintervalls $\benbrace*{t_i,t_{i+1}}$ auf Geraden bewegen.
	Insbesondere ist nämlich der Abstand zwischen zwei Entitäten $x$ und $y$ während $\benbrace*{t_i,t_{i+1}}$ eine konvexe Funktion $\benbrace*{t_i,t_{i+1}} \to \mathbb{R}_+$, wie man sehr leicht mit Hilfe der Dreiecksungleichung zeigt.\footnote{dabei kann man sogar ganz allgemein eine beliebige Norm auf $\mathbb{R}^d$ betrachten}
	Damit sind $x$ und $y$ höchstens während einem Intervall $I \subseteq \benbrace*{t_i,t_{i+1}}$ direkt-zusammenhängend.
	Dieses Intervall liefert wiederum höchstens zwei Vertices in $\mathcal{R}$.
	Da die Trajektorien aus je $\tau$ Kanten bestehen, erzeugt jedes Paar $(x,y)$ somit $\mathcal{O}(\tau)$ Vertices.
	Damit gibt es insgesamt $\mathcal{O}(\tau n^2)$ Vertices und damit auch $\mathcal{O}(\tau n^2)$ Kanten, da der Knotengrad beschränkt ist.
\end{beweis}
% section komplex (end)

