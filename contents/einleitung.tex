%!TEX root = ../trajectory-grouping.tex
Betrachtet man eine Menge von sich bewegenden Entitäten, so haben wir bereits in einigen Vorträgen festgestellt, dass die Analyse von Trajektorien aufgrund der großen Datenmengen prinzipiell herausfordernd ist.
Im Gegensatz zu einigen vorherigen Vorträgen soll es in diesem Vortrag aber nicht darum gehen, die Datenmenge sinnvoll zu reduzieren, sondern Gruppierungseffekte in den gegebenen Trajektorien zu erkennen, um somit ein Bild über die erfassten Bewegungen zu erhalten.
Konkret soll hier eine Modell zur Gruppierung nach \textcite{buchin2015} mit Methoden der Topologie erläutert werden.

Intuitiv soll eine Menge ausreichend vieler sich bewegender Entitäten, die sich für eine gewisse Zeit lang \enquote{zusammen bewegen}, eine Gruppe bilden.
Unsere erste Aufgabe wird es sein, diese Intuition in eine mathematische Definition zu überführen.
Diese intuitive Definition suggeriert bereits, dass dabei drei Parameter in eine Definition mit eingehen sollten:
\begin{description}
	\item[Mindestgröße] Auch wenn man prinzipiell den Fall einer einelementigen Gruppe betrachten kann, sollte eine Gruppe im Allgemeinen eine gewisse Mindestgröße haben, die sich gegebenenfalls aus dem Anwendungsfall ergibt, oftmals aber auch während der Analyse variiert wird.
	So gelangt man zu einer detaillierten Sicht auf Gruppierungen, wenn man die Mindestgröße verringert, da so mehr Gruppen entstehen.
	\item[Dauer/zeitlicher Parameter] Entitäten, die sich nur kurz \enquote{begegnen}, aber nicht zusammen bewegen, sollten nicht unbedingt zu einer Gruppe zusammengefasst werden, was durch die Forderung einer minimalen Dauer der gemeinsamen Bewegung gewährleistet wird.
	Auch hier erhält man eine detailliertere Sicht auf Gruppierungen, wenn man die minimale Dauer geringer ansetzt, da dann im Allgemeinen mehr Gruppen entstehen.
	\item[Räumlicher Parameter] Um von einer gemeinsamen Bewegung sprechen zu können, dürfen die Entitäten offensichtlich nicht zu weit voneinander entfernt sein und wir benötigen einen Parameter für den Abstand zwischen den Entitäten.
	Hier erhalten wir eine detailliertere Sicht mit mehr Gruppen durch Vergrößerung dieses Parameters.
\end{description}
Diese drei Parameter ermöglichen es uns, Gruppierungen in verschiedenen Größenordnungen zu analysieren und unser Augenmerk auf bestimmte Gruppen zu lenken: Bei einer hohen Mindestgröße interessieren wir uns nur für große Gruppen, wählen wir einen großen zeitlichen Parameter, so betrachten wir lediglich lange bestehende Gruppen und ein kleiner räumlicher Parameter filtert lose zusammenhängende Gruppen heraus.

Das hier vorgestellte Modell leistet aber noch deutlich mehr als die Identifikation der Gruppen, den es beinhaltet auch die Veränderungen der \GrpStruktur über den betrachteten Zeitraum.
Insbesondere werden neben \emph{Start}- und \emph{End}-Ereignissen auch \emph{Merge}- und \emph{Split}-Ereignisse beachtet, das heißt es lässt sich eine detaillierte Historie der \GrpStruktur erstellen.
Darüber hinaus zeichnet sich das Modell durch folgende Aspekte aus:
\begin{itemize}
	\item Die Definition einer \enquote{Gruppe} ist intuitiv, siehe \cref{cha:def_gruppe}.
	\item Die gerade erwähnten Änderungen der Gruppen werden zu kontinuierlichen Zeitpunkten erfasst, das heißt sie werden auch erkannt, wenn sie \emph{nicht} an den durch die Trajektorien gegebenen Zeitpunkten auftreten (was auch im Allgemeinen der Fall ist).
	\item Die Berechnung ist -- in Relation zur Ausgabegröße -- effizient, siehe \cref{cha:berechnung}.
	\item Die oben erwähnte Variation der Parameter ermöglicht es, Gruppierung auf unterschiedlichen Maßstäben zu untersuchen.
	\item Das Modell kann \enquote{robust} gegenüber Störungen gemacht werden.
\end{itemize}
Diese Eigenschaften des hier vorgestellten Modells sorgen dafür, dass es einigen anderen Modellen zur Gruppierung von Trajektorien überlegen ist. 
Insbesondere das Einbeziehen von Merge- und Split-Ereignissen \emph{und} die kontinuierlichen Zeitpunkte sorgen laut \textcite{buchin2015} dafür, dass das Modell zum Beispiel dem Ansatz mit \emph{flocks} von \textcite{benkert_flocks} überlegen ist, den wir im vorherigen Vortrag kennengelernt haben.

Wir werden nun im folgenden \cref{cha:def_gruppe} Gruppen mathematisch präzise definieren, bevor wir in \cref{cha:reeb_graphen} dann nach einen kurzen Ausflug in die Differentialtopologie den sogenannten Reeb-Graph definieren können, der neben den Gruppen der Hauptbestandteil der \enquote{\GrpStruktur} wird.
Außerdem wird in diesem Kapitel die Komplexität der \GrpStruktur bestimmt.
In \cref{cha:berechnung} wird daraufhin die Berechnung der \GrpStruktur detailliert erläutert; neben dem Ausflug in die Differentialtopologie ist dies der Abschnitt, der sich am meisten von dem vorliegenden Artikel von \textcite{buchin2015} unterscheidet.
Im letzten Kapitel findet eine kurze Evaluation des Modells statt und es werden noch einige Verbesserungen des vorgestellten Modells erläutert ohne dabei aber ins Detail zu gehen. 
