%!TEX root = ../trajectory-grouping.tex
Betrachtet man eine Menge von sich bewegenden Entitäten, so haben wir bereits im dritten Beitrag dieses Seminars festgestellt, dass für die Analyse solcher Daten eine Gruppierung einzelner Trajektorien äußerst nützlich ist.
In diesem Beitrag soll eine solche Gruppierung mit Methoden der Topologie erläutert werden.\todo{Kapitel aus Buch durchlesen, um Unterschiede betonen zu können}

Intuitiv soll eine Menge ausreichend vieler sich bewegender Entitäten, die sich für eine gewisse Zeit lang \enquote{zusammen bewegen}, eine Gruppe bilden.
Unsere erste Aufgabe wird es sein, diese Intuition in eine mathematische Definition zu überführen.


\todo[inline]{mögliche Anwendungsgebiete}