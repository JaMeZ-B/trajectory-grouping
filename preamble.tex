%!TEX root = trajectory-grouping.tex
%!TEX TS-options = -shell-escape
%!BIB programm = biber
% -- Author: Jannes Bantje, j.bantje@wwu.de
\documentclass[%
	paper=a4,
	fontsize=10,
	DIV=13,
	BCOR=9mm,
	cleardoublepage=empty,
	headinclude,
	footinclude,
	headheight=20.7pt,
	footheight=18pt,
	headings=optiontohead,
	toc=bibliography,toc=listof,
	index=totoc,
	ngerman,
	draft=true
]{scrreprt}
% \usepackage{scrpage2} % wie fancyhdr, nur optimiert auf KOMA-Skript, leicht andere Syntax


%-- Zum "Programmieren"
% ======================================================================================
\usepackage{etoolbox,letltxmacro} % zum Programmieren
% ======================================================================================


%-- Für Farben und Grafiken allgemein
% ======================================================================================
\usepackage[x11names]{xcolor}
\definecolor{dark_gray}{gray}{0.45}
\definecolor{light_gray}{gray}{0.6}
\definecolor{fb10_blue}{cmyk}{0.8,0.4,0.13,0.07}
\usepackage[final]{graphicx}
\robustify\rotatebox
% ======================================================================================



%-- typografische Verbesserungen, Codierungskram, Schriftwahl und erste Mathepakete
% ======================================================================================
\usepackage[utf8]{inputenc}
\usepackage{ccfonts}
% \usepackage[semibold]{sourcesanspro}
\usepackage[semibold,scale=.915]{montserrat}
\usepackage[semibold,scale=.95]{sourcecodepro}
\usepackage{fontawesome}
\usepackage[T1]{fontenc}
\usepackage{textcomp}
\usepackage[newcommands]{ragged2e} % besserer Flattersatz

\usepackage{mathtools} % beinhaltet amsmath
\usepackage{eulervm}
\usepackage{babel}
\usepackage[babel=true,final,tracking=smallcaps]{microtype}
\SetTracking{ encoding = *, shape = sc}{ 45 } % aktiviert ein nicht übertriebenes Tracking von SmallCaps
\DisableLigatures{encoding = T1, family = tt* } % deaktiviere alle Ligaturen für Mono-Spacing-Fonts
% ======================================================================================


%-- Mathepakete 
% ======================================================================================
\mathtoolsset{centercolon,showmanualtags}
\newtagform{brackets}{[}{]}
\usetagform{brackets}
\usepackage{amssymb} % zusätzliche Symbole
\usepackage{xfrac}
\usepackage{mathdots} % Verbesserung von Punkten wie zB \ldots
\usepackage[bb=px]{mathalfa}
% ======================================================================================


%-- Hässliche Hacks, um FontWarnings loszuwerden
% ======================================================================================
\usepackage{silence}
\WarningFilter{latexfont}{Size substitutions with differences}
\WarningFilter{latexfont}{Font shape `U/bbold/m/n' in size}
% ======================================================================================

%!TEX root = trajectory-grouping.tex

% -- Zum Finetuning von Befehlen
\makeatletter
\newcommand{\raisemath}[1]{\mathpalette{\raisem@th{#1}}}
\newcommand{\raisem@th}[3]{\raisebox{#1}{$#2#3$}}
\makeatother

% -- Box mit vorgegebener minimaler Länge
\DeclareRobustCommand{\minwidthbox}[2]{%
  \ifmmode
    \expandafter\mathmakebox
  \else
    \expandafter\makebox
  \fi
  [\ifdim#2<\width\width\else#2\fi]{#1}%
}

% -- farbiges Untersteichen im Mathe-Modus
\def\mathul#1#2{\color{#1}\underline{{\color{black}#2}}\color{black}} 


% -- besserer underbrace befehl
\newcommand{\Underbrace}[2]{{\underbrace{#1}_{#2}}}





%--Mengen
\newcommand\SetSymbol[1][]{\nonscript\:#1\vert\allowbreak\nonscript\:\mathopen{}}
\providecommand\given{} % to make it exist
\DeclarePairedDelimiterX\set[1]\{\}{\renewcommand\given{\SetSymbol[\delimsize]}#1}

% -- Betrag und Skalarprodukt
\DeclarePairedDelimiter{\abs}{\lvert}{\rvert}
\DeclarePairedDelimiterX\skal[2]{\langle}{\rangle}{#1\,,\,#2}

%--Umklammern
\DeclarePairedDelimiter\enbrace{(}{)}
\DeclarePairedDelimiter\benbrace{[}{]}
\DeclarePairedDelimiter\homo{\llbracket}{\rrbracket}
\newcommand{\ssbrace}[1]{{\scriptscriptstyle\enbrace{#1}}}

%--Norm
\DeclarePairedDelimiter\norm{\Vert}{\Vert}






% - verbessertes Integral
\newcommand{\Int}[1]{\int_{\mathrlap{#1}}\,\,}

% -- Alles spezielles aus der Differentialtopologie
\newcommand{\Tang}{\ensuremath{\mathrm{T}\mkern-0.85mu}}
\newcommand{\mathd}{\ensuremath{\mathrm{d}\mkern-0.7mu}}
\newcommand{\diff}[2]{\ensuremath{\frac{{\partial #1}}{{\partial #2}}}}
\newcommand{\diffs}[2]{\ensuremath{\partial #1/\partial #2}}
\newcommand{\diffd}[2]{\ensuremath{\frac{\mathd #1}{\mathd #2} }}
\DeclareMathOperator{\Lie}{L}
\DeclareMathOperator{\Ad}{Ad}  
\DeclareMathOperator{\ad}{ad}
\DeclareMathOperator{\Spin}{Spin}
% \DeclareMathOperator{\spin}{spin}
\newcommand{\spin}{\mathop{\mathfrak{spin}}}
\newcommand{\Ce}{\mathcal{C}}
\DeclareMathOperator{\vol}{vol}



%--Abbildungsdefinition
\newcommand{\mapdef}[5]{%
	\[
		\begin{array}{rcl}
			\textstyle #1 &\xrightarrow{\minwidthbox{#5}{2em}} & \textstyle #2 \\[0.5ex]
			\textstyle #3 &\xmapsto{\minwidthbox{\mbox{ }}{2em}} & \textstyle #4
		\end{array}
	\]
}

%--modifiziertes Stackrel
\newcommand{\StackText}[2]{\stackrel{\mbox{\scriptsize #1}}{#2}}
\newcommand{\StackTextClap}[2]{\stackrel{\mathclap{\mbox{\scriptsize #1}}}{#2}}



 % Liste mit Mathebefehlen laden

% -- TikZ-Kram
% ======================================================================================
\usepackage{tikz-cd}
\usetikzlibrary{quotes,babel,arrows.meta}
\tikzset{>=Straight Barb[]}
\tikzcdset{arrow style=math font}
\usetikzlibrary{external}
\tikzexternalize[prefix=Bilder/tikz/, up to date check=diff]
\tikzset{external/system call={xelatex \tikzexternalcheckshellescape %-- verwende LuaLaTeX, wegen dynamischer Speicherallokation
    -halt-on-error -interaction=batchmode -jobname "\image" "\texsource"}}
\AtBeginEnvironment{tikzcd}{\tikzexternaldisable}
\AtEndEnvironment{tikzcd}{\tikzexternalenable}
\pgfkeys{/pgf/images/include external/.code=\includegraphics{#1}}
% ======================================================================================


%-- Kopf- und Fußzeilen und allgemeine KOMO-Skript Formatierung
% ======================================================================================
\addtokomafont{chapter}{\Huge}
\addtokomafont{captionlabel}{\sffamily\fontseries{mb}\selectfont}
\usepackage[%
	headsepline=1pt,
	automark,
	draft=false,
	]%
{scrlayer-scrpage}
\pagestyle{scrheadings}
\clearscrheadfoot % Standardkonfiguration löschen
\ohead{\includegraphics[height=0.6 cm,keepaspectratio]{Bilder/wwulogo.pdf}}
\ihead{\rule{0cm}{0.6cm}\headmark}
\ofoot*{\Large\sffamily\thepage}
\ifoot{\color{light_gray}Gruppierung von Trajektorien -- Seminarvortrag von Jannes Bantje}
% ======================================================================================

%-- Satzspiegel neu berechnen
% ======================================================================================
\flushbottom
\KOMAoptions{DIV=last}
% ======================================================================================

% -- BibLaTeX 
% ======================================================================================
\usepackage[%
	backend=biber,
	sortlocale=auto,
	natbib,
	hyperref,
	backref,
	style=alphabetic
	]%
{biblatex}
\renewcommand*{\mkbibnamelast}[1]{%
  \ifmknamesc{\textsc{#1}}{#1}}

\renewcommand*{\mkbibnameprefix}[1]{%
  \ifboolexpr{ test {\ifmknamesc} and test {\ifuseprefix} }
    {\textsc{#1}}
    {#1}}

\def\ifmknamesc{%
  \ifboolexpr{ test {\ifcurrentname{labelname}}
               or test {\ifcurrentname{author}}
               or ( test {\ifnameundef{author}} and test {\ifcurrentname{editor}} ) }}
\addbibresource{quellen.bib}
\DefineBibliographyStrings{ngerman}{%
  andothers = {et\addabbrvspace al\adddot}
}
% ======================================================================================


% -- Konfiguration von Hyperref und Cleveref
% ======================================================================================
\usepackage[hidelinks, pdfpagelabels, bookmarksopen=true, bookmarksnumbered=true, linkcolor=black, urlcolor=SkyBlue2, plainpages=false,pagebackref, citecolor=black, hypertexnames=true, pdfauthor={Jannes Bantje}, pdfborderstyle={/S/U}, linkbordercolor=SkyBlue2, colorlinks=false,backref=false]{hyperref}
\hypersetup{final}
\usepackage[nameinlink,noabbrev]{cleveref}
\newcommand{\appendLink}[1]{#1\,\faExternalLink}
\newcommand{\hrefsym}[2]{\href{#1}{\texttt{\appendLink{#2}}}}
\renewcommand{\url}[1]{\hrefsym{#1}{\nolinkurl{#1}}}
% ======================================================================================

%-- Schöne Emailadressen setzen (nach http://tex.stackexchange.com/a/663/67086)
% ======================================================================================
\catcode`\_=11\relax
\newcommand\email[1]{\_email #1\q_nil}
\def\_email#1@#2\q_nil{%
  \href{mailto:#1@#2}{{\emailfont #1\emailampersat #2}}
}
\newcommand\emailfont{\ttfamily}
\newcommand\emailampersat{{\sffamily@}}
\catcode`\_=8\relax
% ======================================================================================

% -- marginnotes/todonotes 
% ======================================================================================
\usepackage[fulladjust]{marginnote}
\renewcommand*{\marginfont}{\itshape\footnotesize}
\renewcommand*{\raggedleftmarginnote}{\RaggedLeft}
\renewcommand*{\raggedrightmarginnote}{\RaggedRight}
\usepackage[textsize=small,obeyDraft]{todonotes}
\LetLtxMacro{\oldtodo}{\todo}
\renewcommand{\todo}[2][]{\tikzexternaldisable\oldtodo[#1]{#2}\tikzexternalenable}
\LetLtxMacro{\oldmissingfigure}{\missingfigure}
\renewcommand{\missingfigure}[2][]{\tikzexternaldisable\oldmissingfigure[{#1}]{#2}\tikzexternalenable}
\deffootnote[1.5em]{1.5em}{1.5em}{\textsuperscript{\thefootnotemark}\ }
% ======================================================================================


%-- Für Auflistungen und Tabellen
% ======================================================================================
\usepackage{multicol}
\usepackage[shortlabels,inline]{enumitem}
\setlist{itemsep=1pt,parsep=2pt,topsep=2pt}
\setlist[itemize,1]{label=\textcolor{black}{\raisebox{.45ex}{\rule{.6ex}{.6ex}}}}
\setlist[enumerate]{font=\sffamily\fontseries{mb}\selectfont}
\usepackage{booktabs}
% ======================================================================================


%-- Für Abbildungen
% ======================================================================================
\usepackage[caption=false]{subfig}
\captionsetup[subfigure]{subrefformat=simple,labelformat=simple,listofformat=subsimple}
\renewcommand\thesubfigure{(\alph{subfigure})}
\usepackage{wrapfig}
% ======================================================================================


%-- Inhaltsverzeichnis konfigurieren
% ======================================================================================
\usepackage[tocindentauto]{tocstyle}
\usetocstyle{KOMAlike}
% ======================================================================================




% -- Indexverarbeitung
% ======================================================================================
\usepackage{makeidx}
\newcommand{\bet}[1]{\emph{#1}}
\newcommand{\Index}[1]{\bet{#1}\index{#1}}
\makeindex
\renewcommand{\indexpagestyle}{scrheadings}
% ======================================================================================


% -- theorem packages
% ======================================================================================
\usepackage{amsthm}
\usepackage{thmtools,thm-restate}
\usepackage{mdframed}
\usepackage{bookmark}
\bookmarksetup{open,numbered}
\makeatletter
\newcommand*{\theorembookmark}{%
  \bookmark[
    dest=\@currentHref,
    rellevel=1,
    keeplevel,
  ]{%
    \thmt@thmname\space\csname the\thmt@envname\endcsname
    \ifx\thmt@shortoptarg\@empty
    \else
      \space(\thmt@shortoptarg)%
    \fi
  }%
}   
\makeatother
% -- Definition der einzelnen Umgebungen
\declaretheoremstyle[%
	headfont=\sffamily\bfseries,
	notefont=\normalfont\sffamily,
	bodyfont=\normalfont,
	headformat=\NUMBER\ \NAME\NOTE,
	headpunct=.,
	postheadspace=1em,
	spaceabove=15pt,spacebelow=10pt,
	shaded={bgcolor=gray!20},
	postheadhook=\theorembookmark]%
{mainstyle}
\declaretheoremstyle[%
	headfont=\sffamily\bfseries,
	notefont=\normalfont\sffamily,
	bodyfont=\normalfont,
	headformat=\NUMBER\ \NAME\NOTE,
	headpunct=.,
	postheadspace=1em,
	spaceabove=15pt,spacebelow=10pt,
	shaded={bgcolor=fb10_blue!20},
	postheadhook=\theorembookmark]%
{mainstyle_blue}
\declaretheoremstyle[%
	headfont=\sffamily\bfseries,
	notefont=\normalfont\sffamily,
	bodyfont=\normalfont,
	headformat=\NUMBER\ \NAME\NOTE,
	headpunct=.,
	postheadspace=1em,
	spaceabove=15pt,spacebelow=10pt,
	postheadhook=\theorembookmark]%
{mainstyle_unshaded}
\declaretheoremstyle[%
	headfont=\sffamily\bfseries,
	notefont=\normalfont\sffamily,
	bodyfont=\normalfont,
	headformat=swapnumber,
	headpunct=.,
	postheadspace=1em,
	spaceabove=15pt,spacebelow=10pt,
	shaded={bgcolor=gray!20},
	postheadhook=\theorembookmark,
	qed=\qedsymbol]%
{mainstyleB}
\declaretheorem[name=Definition,parent=chapter,style=mainstyle_blue]{definition}
\declaretheorem[name=Theorem,sharenumber=definition,style=mainstyle]{theorem}
\declaretheorem[name=Proposition,sharenumber=definition,style=mainstyle,refname=Proposition]{proposition}
\declaretheorem[name=Lemma,sharenumber=definition,style=mainstyle]{lemma}
\declaretheorem[name=Satz,refname=Satz,sharenumber=definition,style=mainstyle]{satz}
\declaretheorem[name=Korollar,sharenumber=definition,style=mainstyle]{korollar}
\declaretheorem[name=Bemerkung,sharenumber=definition,style=mainstyle_unshaded]{bemerkung}
% -- Beweise
\declaretheoremstyle[headfont=\bfseries\scshape,bodyfont=\normalfont,headpunct=:,postheadspace=1em,spacebelow=12pt,spaceabove=2pt,qed=\qedsymbol]{beweise}
\declaretheoremstyle[headfont=\bfseries\scshape,bodyfont=\normalfont,headpunct=:,postheadspace=1em,spacebelow=12pt,spaceabove=2pt]{beweisskizze}
\declaretheorem[name=Beweis,numbered=no,style=beweise]{beweis}
% ======================================================================================


% -- Inhaltsverzeichnis und weitere Pakte, die zuletzt geladen werden sollten
% ======================================================================================
\usepackage[autostyle,german=quotes]{csquotes}
\usepackage{ellipsis}
% ======================================================================================
